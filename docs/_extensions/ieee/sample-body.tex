% This is minimum body for a paper with one fig, one table, one ref
\section{Introduction}
\lipsum[1]

\begin{mytable}
\begin{longtable}[]{@{}llllll@{}}
\caption{\label{tbl-parameters}Table of sensor
parameters}\tabularnewline
\toprule()
Parameter & Symbol & Min & Typ & Max & Unit \\
\midrule()
\endfirsthead
\toprule()
Parameter & Symbol & Min & Typ & Max & Unit \\
\midrule()
\endhead
Supply current & \(i_\mathrm{off}\) & - & - & 10 & mA \\
Hall sensitivity & \(S_\mathrm{H}\) & 0.2 & - & - & V/T \\
Effective nr. of bits & \(\mathrm{ENOB}\) & 12 & - & - & - \\
\bottomrule()
\end{longtable}
\end{mytable}

\lipsum[2]

% figure* for 2-column figure
\begin{figure}[htbp]
\centerline{\includegraphics[width=2in]{example-image}}
\caption{Example of a figure caption.}
\label{fig}
\end{figure}

\begin{thebibliography}{00}
\bibitem{1} T. Le Signor, N. Dupre, and G. F. Close, {``A
gradiometric magnetic force sensor immune to stray magnetic fields for
robotic hands and grippers,''} \emph{IEEE Robotics and Automation
Letters}, vol. 7, no. 2, pp. 3070--3076, Apr. 2022, doi:
\href{https://doi.org/10.1109/lra.2022.3146507}{10.1109/lra.2022.3146507}.
\end{thebibliography}